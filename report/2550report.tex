% -----
% STANDARD TeX TEMPLATE
% CHRISTOPHER CLAOUE-LONG
% -----
% -
% - DOCUMENT GEOMETRY SETUP
\documentclass[12pt,a4paper]{article}
\usepackage{geometry}
\geometry{margin=20mm}
\usepackage{lastpage}
\makeatletter \renewcommand{\@oddfoot}{\hfil Page \thepage\ of \pageref{LastPage} \hfil} \makeatother
% -
% - SECTION FORMATTING
\renewcommand \thepart{\Roman{part}}
\renewcommand \thesection{\arabic{section}}
\renewcommand \thesubsection{\arabic{section}.\arabic{subsection}}
\renewcommand \thesubsubsection{\arabic{section}.\arabic{subsection}.\arabic{subsubsection}}
% -
% - FONT
\usepackage{amsmath, amsthm, amssymb,graphicx,epstopdf}
\DeclareGraphicsRule{.tif}{png}{.png}{`convert #1 `dirname #1`/`basename #1 .tif`.png}
\usepackage[sc]{mathpazo}
\linespread{1.05}
\usepackage[T1]{fontenc}
\usepackage[bitstream-charter]{mathdesign}
% -
% - MISC. PACKAGES
\usepackage{color}
\usepackage[usenames,dvipsnames,svgnames,table]{xcolor}\usepackage{tikz} \usepackage{qtree, tikz-qtree, lineno}
\renewcommand\linenumberfont{\normalfont\sffamily}
\usepackage{datetime, multicol, verbatim, ulem, alltt, multirow, hyperref}
\hypersetup{
colorlinks,
citecolor=black,		% - Citation colour
filecolor=black,		% - File colour
linkcolor=black,		% - Link colour
urlcolor=black		% - URL colour
}
\urlstyle{same}
% -
% -
% - MISC. SYMBOLS AND COMMANDS
% - Thick horizontal blue line
\newcommand{\Hrule}{\textcolor{blue}{\rule{\linewidth}{0.5mm}}}
% -----
\begin{document}
% -----
% - Title
\begin{center}
\Hrule\\
\textbf{\Huge COMP2550/COMP3130\\Warmup Project Report}\\
\textbf{\large Christopher Claou\'e-Long and Jimmy Lin}\\
\Hrule
\end{center}
% -
% -
\begin{multicols}{2}
\section{Overview}
The task at hand for this warmup project was to implement a set of useful features for a computer vision task, using the open source Darwin framework for machine learning and the MSRC dataset.  The desired outcome was a feature vector that allowed the machine learning algorithms to classify images with over 50\% accuracy.

\section{Method}

\hspace{4mm} First of all, we consider the texture of one superpixel. One possible implementation is to respectively discretize the lightness of marginal RGB value (9 for each interval length) and assign each pixel to those intervals based on their RGB lightness. The strength of this method is the tremendous informativeness and it does result in a satisfying accuracy increase (directly to arround 0.43) . However, that discretization method occupies too many dimensions of feature vector (87 attributes) to the extent that the classifier was badly sensitive to other features. Alternatively,we attempt the mean and standard deviation of absolute marginal RGB value (only 6 attributes). Expectedly, it produces a considerable rise with occupying fewer dimension of feature vector. (It will be demonstrated latter why occupation of less dimensions is vital.) 

The second significant set of features is the mean and standard deviation of relative lightness of one pixel's marginal RGB value, that is, the lightness difference between the pairs of Red and Green, Green and Blue, Blue and Red. 

On top of that, some other features that lead to comparatively slight accuracy advances are added into the our feature vector as well. For instance, the average smoothness counts the mean of RGB lightness difference of all pixels to their neighbour pixels. And location of one superpixel, more specifically, evalutates the mean and standard deviation of x,y coordinate of one pixel in its located image. The idea of adding this set of feature our final selection comes from the semantic perceptions that the higher stuff is more likely to be sky and that the lower stuff is more likely be ground or grass. 

What we discussed before is all about the features of one superpixel itself. To further enhance the accuracy, we may need to, when forming the feature vector of one superpixel, take into consideration the brief information of neighboured superpixel. That is an intuitively meaningful approach because even human cannot recognize a small region without perceiving its surroundings. Yet, the experiments we have accomplished till now does not indicate the effectiveness of that approach. One possible reason is that the features we pick out for this approach provide little information regarding the surrounded environment of one superpixel. By the way, we apply some algorithm optimizations to reduce the time cost for training of classifier with neighbour superpixel detection, from about 6 hours to 25 minutes. One way to fulfill this progress is to replace linear list with hash set to minimize the time for membership evaluation. 

\section{Results}
Accuracy tables with different features implemented.\\

\section{Discussion}
Interpretation of results, what went well, what went wrong (overfitting?), what could be done better. \\

\section{References}
S. Gould. DARWIN: A Framework for Machine Learning and Computer Vision Research and Development. In \textit{Journal of Machine Learning Research (JMLR)}, 2012.\\
K. Park and S. Gould. On Learning Higher-Order Consistency Potentials for Multi-class Pixel Labeling. In \textit{Proceedings of the European Conference on Computer Vision (ECCV)}, 2012.

\end{multicols}
\vfill\Hrule
\end{document}
% -----
% END OF LINE
% -----
