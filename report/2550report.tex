% -----
% STANDARD TeX TEMPLATE
% CHRISTOPHER CLAOUE-LONG
% -----
% -
% - DOCUMENT GEOMETRY SETUP
\documentclass[11pt,a4paper]{article}
\usepackage{geometry}
\geometry{margin=20mm}
\usepackage{lastpage}
\makeatletter \renewcommand{\@oddfoot}{\hfil Page \thepage\ of \pageref{LastPage} \hfil} \makeatother
% -
% - SECTION FORMATTING
\renewcommand \thepart{\Roman{part}}
\renewcommand \thesection{\arabic{section}}
\renewcommand \thesubsection{\arabic{section}.\arabic{subsection}}
\renewcommand \thesubsubsection{\arabic{section}.\arabic{subsection}.\arabic{subsubsection}}
% -
% - FONT
\usepackage{amsmath, amsthm, amssymb,graphicx,epstopdf}
\DeclareGraphicsRule{.tif}{png}{.png}{`convert #1 `dirname #1`/`basename #1 .tif`.png}
\usepackage[sc]{mathpazo}
\linespread{1.05}
\usepackage[T1]{fontenc}
\usepackage[bitstream-charter]{mathdesign}
% -
% - MISC. PACKAGES
\usepackage{color}
\usepackage[usenames,dvipsnames,svgnames,table]{xcolor}\usepackage{tikz} \usepackage{qtree, tikz-qtree, lineno}
\renewcommand\linenumberfont{\normalfont\sffamily}
\usepackage{datetime, multicol, verbatim, ulem, alltt, multirow, hyperref}
\hypersetup{
colorlinks,
citecolor=black,		% - Citation colour
filecolor=black,		% - File colour
linkcolor=black,		% - Link colour
urlcolor=black		% - URL colour
}
\urlstyle{same}
% -
% -
% - MISC. SYMBOLS AND COMMANDS
% - Thick horizontal blue line
\newcommand{\Hrule}{\textcolor{blue}{\rule{\linewidth}{0.5mm}}}
\newcommand{\HUGEBOLD}[1]{\textbf{\Huge{#1}}}
% -----
\begin{document}
% -----
% - Title
\begin{center}
\Hrule\\
\textbf{\Huge COMP2550/COMP3130 ANU\\Warmup Project Report}\\
\textbf{\\\large Christopher Claou\'e-Long 
(\href{mailto:u5183532@anu.edu.au}{\textit{\underline{\smash{u5183532@anu.edu.au}}}})\\
Jimmy Lin 
(\href{mailto:yourUIDhere@anu.edu.au}{\textit{\underline{\smash{yourUIDhere@anu.edu.au}}}})\\}
\Hrule
\end{center}
% -
% -
\begin{multicols}{2}
\section{Overview}
The task at hand for this warmup project was to implement a set of useful features for a computer vision task, using the open source Darwin framework for machine learning and the MSRC dataset.  The desired outcome was a feature vector that allowed the machine learning algorithms to classify images with over 50\% accuracy.

\section{Method}
The first few features we thought to implement were of course to measure the average red, green and blue values of a superpixel, since this would allow us to start differentiating between extremely different superpixels.  Adding a standard deviation of the pixels' colours compared to the average colours over the superpixel would also help differentiate further between superpixels of radically different texture/contrast.  We also tried out getting the difference and absolute difference between the red, green and blue average values across the superpixel to further bring out features dependent on the superpixel composition. \HUGEBOLD{CONTINUE HERE.}\\\\
Intuition, thoughts and discussion on the CV features\\
RGB values and std deviation, centre (position), average grayscale, difference between colours in a superpixel, neighbours, texture/contrast etc.

\section{Results}
Pictures, maybe accuracy tables with different features implemented?

\section{Discussion}
numerical analysis of your algorithm - emailed Stephen for clarification on this.\\
What is the performance of your algorithm on the training set compared to the test set? Is this result expected?\\
Why is it important to evaluate pixelwise accuracy instead of accuracy on the superpixels?\\
What do you think is more important, the features or the machine learning classifier?\\
Interpretation of results, what went well, what went wrong (overfitting?), what could be done better.

\section{References?}
Do we have any?
\end{multicols}
\vfill\Hrule
\end{document}
% -----
% END OF LINE
% -----